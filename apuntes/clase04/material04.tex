\begin{framed}

Objetivos:
\begin{itemize}
    \item Estudiar la superposición de flujos potenciales. 
\end{itemize}

Contenidos:
\begin{itemize}
    \item Repaso de flujos elementales.
    \item El principio de superposición en la ecuación de Laplace. 
    \item Ejemplos de aplicación
    \begin{itemize}
        \item El cuerpo de Rankine. 
        \item Flujo alrededor de un cilindro. 
    \end{itemize}
\end{itemize}

Bibliografía:
\begin{itemize}
    \item White, F. M. (2008) Mecánica de Fluidos. McGraw-Hill. Sexta edición. Secciones 4.10
    \item Fox, R. W., Pritchard, P. J. y McDonald, A. T. (2009) Introduction to Fluid Mechanics. John Wiley \& Sons. Sección 6.7.
\end{itemize}
\end{framed}

\section*{Flujos elementales}

La clase pasada estudiamos flujos potenciales elementales.
Estos son, básicamente, soluciones analíticas de la ecuación de Laplace (con un $\delta$ de Dirac en algunos casos) con condiciones de contorno que tienen sentido físico.
Para el caso de los flujos elementales que estudiamos, la condición de que usamos es de velocidad hacia el infinito, que era $U_\infty$ en el flujo uniforme, y cero en los vórtices, fuente, sumidero y dobletes.
Sin embargo, no hay que confundirse: cualquier solución de la ecuación de Laplace que cumpla con condiciones de borde físicas representa el movimiento de un flujo ideal.
De hecho, hoy vamos a encontrar más soluciones de flujo potencial.

Solo como recordatorio de la clase pasada, los flujos elementales son:

\begin{itemize}
\item Flujo uniforme horizontal (dirección $+x$):
\begin{align}
u &= U_\infty \quad v = 0 \nonumber \\
\phi&=\pm U_\infty x \quad \psi= \pm U_\infty y
\end{align}

\item Flujo uniforme vertical (dirección $+y$):
\begin{align}
u &= 0 \quad v = V_\infty \nonumber \\
\phi&=\pm V_\infty y \quad \psi= \mp V_\infty x
\end{align}

\item Fuente:
\begin{align}
V_r &= \frac{q}{2\pi r}\quad V_\theta=0\nonumber\\
\phi&=\frac{q}{2\pi}\ln(r) \quad \psi=-\frac{q}{2\pi}\theta
\end{align}

\item Sumidero:
\begin{align}
V_r &= -\frac{q}{2\pi r}\quad V_\theta=0\nonumber\\
\phi&=-\frac{q}{2\pi}\ln(r) \quad \psi=\frac{q}{2\pi}\theta
\end{align}

\item Vórtice (dirección contra reloj):
\begin{align}
V_r &= 0 \quad V_\theta=\frac{\Gamma}{2\pi r}\nonumber\\
\phi&=\frac{\Gamma}{2\pi}\theta \quad \psi = -\frac{\Gamma}{2\pi}\ln(r).
\end{align}

\item Doblete: 
\begin{align}
V_r = -K\frac{\cos(\theta)}{r^2}\quad V_\theta = -K\frac{\sin(\theta)}{r^2}\nonumber\\
\phi=K\frac{\cos(\theta)}{r}\quad \psi = K\frac{\sin(\theta)}{r}.
\end{align}

\end{itemize}

\section*{Principio de superposición}
Sabemos que para flujos ideales tanto $\phi$ como $\psi$ satisfacen la ecuación de Laplace.
La ecuación de Laplace es homogenea (todos los términos de la ecuación contienen la variable dependiente) y lineal (la variable dependiente no está elevada a una potencia), por lo tanto, si $\phi_1$ y $\phi_2$ satisfacen la ecuación de Laplace, $\phi_3=\phi_1+\phi_2$ también es una solución.
Si aterrizamos esto al caso de flujo potencial, podríamos decir que $\mathbf{V}_3=\mathbf{V}_1+\mathbf{V}_2$, además $\psi_3=\psi_1+\psi_2$.
Esta propiedad nos permite combinar soluciones simples de flujo potencial, como los flujos elementales, para modelar flujos más interesantes para aplicaciones reales.
De hecho, esta ya lo hicimos sin darnos cuenta cuando realizamos la derivación del doblete: sumamos la solución de una fuente y un sumidero.
En esta clase, vamos a revisar algunas soluciones con superposición que son relevantes en mecánica de fluidos.

Algo que no hemos discutido en profundidad todavía, y que hoy tomará importancia, es la interpretación de líneas de flujo como cuerpos sólidos.
La característica de las líneas de flujo es que la velocidad es tangencial a ellas, por lo tanto, no existe flujo ``a través'' de una línea de flujo.
Por otra parte, al ser este un flujo no viscoso, la condición de contorno correspondiente es la de impermeabilidad (flujo no pasa a través de un sólido, $\mathbf{V}\cdot\mathbf{n}=0$), pero si puede resbalar por él.
Si se fijan, la condición de borde de impermeabilidad es equivalente a una línea de flujo, por lo que podemos mirar a una línea de flujo como un cuerpo sólido.
Vamos a revisitar esto en los ejemplos que veremos a continuación, y les quedará mucho más claro.
